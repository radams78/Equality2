\documentclass{easychair}
<<<<<<< HEAD

\begin{document}
Text
\end{document}
=======
\bibliographystyle{plain}

\title{A Type Theory with Native Homotopy Universes}
\author{Robin Adams\inst{1} \and Andrew Polonsky\inst{2}}
\institute{Universitetet i Bergen \and University Paris Diderot}

\usepackage{proof}

\newcommand{\Prop}{\ensuremath{\mathbf{Prop}}}
\newcommand{\Set}{\ensuremath{\mathbf{Set}}}
\newcommand{\Groupoid}{\ensuremath{\mathbf{Groupoid}}}
\newcommand{\LEtwo}{\ensuremath{\lambda \simeq_2}}
\newcommand{\eqdef}{\mathrel{\smash{\stackrel{\text{def}}{=}}}}

\begin{document}
\maketitle

We present a type theory $\LEtwo$ with an extensional equality relation; that is, the universe of types is closed by reflection into it of the logical relation defined by induction on the structure of types.

The type system has four universes:
\begin{itemize}
\item
The trivial universe $\mathbf{1}$, with one type $\top$ that has one object $*$.
\item
The universe $\Prop$ of \emph{propositions}.  An object of $\Prop$ is called a \emph{proposition}, and the objects of a proposition are called \emph{proofs}.
\item
The universe $\Set$ of \emph{sets}.
\item
The universe $\Groupoid$ of \emph{groupoids}.
\end{itemize}

For each universe $U$, we have an associated relation of equality between the universes $\simeq$, and between objects of the universes $\sim$, with associated rules of deduction:

\[ \infer{A \simeq B : U}{A : U \quad B : U}
\qquad
\infer{a \sim_e b : U^-}{a : A \quad e : A \simeq B \quad b : B} \]

where $U^-$ is the universe one dimension below $U$.  Thus:
\begin{itemize}
\item
Given two propositions $\phi$ and $\psi$, we have the proposition $\phi \simeq \psi$ which denotes the proposition `$\phi$ if and only if $\psi$'.  If $\delta : \phi$, $\epsilon : \psi$ and $\chi : \phi \simeq \psi$, then $\delta \sim_\chi \epsilon = * : \mathbf{1}$.  (\emph{Cf} In homotopy type theory, any two objects of a proposition are equal.)
\item
Given two sets $A$ and $B$, we have the set $A \simeq B$, which denotes the set of all bijections between $A$ and $B$.  Given $a : A$, $f : A \simeq B$ and $b : B$, we have the proposition $a \sim_f b : \Prop$, which denotes that $a$ is mapped to $b$ by the bijection $f$.
\item
Given two groupoids $G$ and $H$, we have the groupoid $G \simeq H$, which denotes the groupoid of all groupoid isomorphisms between $G$ and $H$.  Given $g : G$, $\phi : G \simeq H$ and $h : H$, we have the set $g \sim_\phi h : \Set$, which can be thought of as the set of all paths between $\phi(g)$ and $h$ in $H$.
\end{itemize}

We have reflexivity provided by the following typing rules:
\[ \infer{A : U}{1_A : A \simeq A} \qquad \infer{a : A}{r_a : a \sim_{1_A} a} \]
The relation $\sim_{1_A}$ thus behaves like an equality relation on each type $A$.

Each universe is itself an object of the next universe:
\[ \mathbf{1} : \Prop : \Set : \Groupoid \]
and we have the following definitional equalities:
\[ \phi \sim_{1_\Prop} \psi \eqdef \phi \simeq \psi, \quad
A \sim_{1_\Set} B \eqdef A \simeq B \]

The following computation rules also hold in $\LEtwo$.

\textbf{TODO}

We therefore note the following features of $\LEtwo$:
\begin{itemize}
\item
Univalence holds definitionally --- an equality between types $A \simeq B$ is exactly the type of equivalences between $A$ and $B$.
\item
Transport respects reflexivity and composition definitionally.
\end{itemize}

This type theory has been formalised in Agda, using the method of the system $\mathtt{Kipling}$ from McBride \cite{McBridea}.

The formalisation is available online at https://github.com/radams78/Equality2.

\bibliography{type}

\end{document}
>>>>>>> a206cc8a33ea749bd2322212ad62b14ee5c09062
